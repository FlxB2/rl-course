\documentclass[a4paper]{article}
%\usepackage[ngerman]{babel}
\usepackage[T1]{fontenc}
\usepackage[utf8]{inputenc}
\usepackage{textcomp}
\usepackage{geometry}
\geometry{ left=2cm, right=2cm, top=2cm, bottom=3cm, bindingoffset=5mm}
\usepackage{graphicx}
\usepackage{xcolor}
\usepackage{hyperref}
\usepackage{longtable}
\usepackage{amstext}
\usepackage{array}
\usepackage{amsmath}
\newcolumntype{L}{>{$}l<{$}}
\usepackage{tabularx, ragged2e}
\usepackage{helvet}
\renewcommand{\familydefault}{\sfdefault}
\usepackage{lastpage}
\usepackage{todonotes}
\usepackage{titlesec}
\titleformat*{\section}{\large\bfseries}
\usepackage{listings}
\usepackage{color}

\usepackage{tikz}
%\newcommand{\tikzmark}[2]{\tikz[overlay, remember picture] \node[inner sep=0pt, outer sep=0pt, anchor=base] (#1) {#2};}
\usetikzlibrary{tikzmark}


\definecolor{mygreen}{rgb}{0.18, 0.545, 0.341}
\definecolor{mygray}{rgb}{0.5,0.5,0.5}
\definecolor{myblue}{rgb}{0.53,0.61,0.85}

\lstset{
 keywordstyle=\color{mygreen},
 commentstyle=\color{mygray},
 numbers=left,
 numbersep=5pt, 
 numberstyle=\scriptsize\color{mygray}
 }

\date{}
\author{}
\usepackage{fancyhdr}
\pagestyle{fancy}
\fancyhf{}
\fancyhead[R]{Felix Burk\\ Pascal Huszár}
\fancyhead[L]{Reinforcment Learning \\ Summer Term 2021 }
\fancyfoot[R]{page \thepage \text{ }/ \pageref*{LastPage}}
%\fancyfoot[LE]{Seite \thepage \text{ }von \pageref{LastPage}}
\renewcommand{\headrulewidth}{0.5pt}

\usepackage{amsmath}
\DeclareMathOperator*{\argmax}{arg\,max}
\DeclareMathOperator*{\argmin}{arg\,min}


\title{\textbf{Exercise 01}}

\begin{document}
	\maketitle 
	\thispagestyle{fancy}
	
    \section*{Task 1 - Multi-armed Bandits}
    \begin{itemize}
    \item[a)] $\epsilon$-greedy action selection selects the greedy action with probability $p=1-\epsilon + \frac{\epsilon}{k}$. Therefore, $p=0.75$
    \item[b)]
    	\begin{itemize}
    		\item[1)] Step 2: Because $\argmax_a Q_2(a) = 1 = A_1$ and action 2 is selected \\
    				  Step 5: Because $\argmax_a Q_5(a) = 2 = A_2$ and action 3 is selected \\
    		\item[2)] Step 1: Because $\forall a \in \{1,2,3,4\}: Q_0(a) = 0$ \\
    				  Step 3: Because $Q_2(2)=Q_2(1)=1$ \\
    				  Step 4 is possible as well, because the best action could be picked at random.
    	\end{itemize}
    \vspace{.3in}
    \begin{tabularx}{\linewidth}{c | c | c | l}
    	timestep & action taken & reward received & action-value estimate \\ 
    	0 & - & -  & $Q_0(a) = 0$ \\[2ex]
    	1 & $A_1=1$ & $R_1=1$ & $Q_1(1) = \frac{R_1*1}{1} = 1$ \\[2ex]
    	2 & $A_2=2$ & $R_2=1$ & $Q_2(2) = \frac{R_1*0+R_2*1}{0+1} = 1$ \\[2ex]
    	3 & $A_3=2$ & $R_3=2$ & $Q_3(2) = \frac{R_1*0+R_2*1+R_3*1}{0+1+1} = 3/2$ \\[2ex]
    	4 & $A_4=2$ & $R_4=2$ & $Q_4(2) = \frac{R_1*0+R_2*1+R_3*1+R_4*1}{0+1+1+1} = 5/3$  \\[2ex]
    	5 & $A_5=3$ & $R_5=0$ & $Q_5(3) = \frac{R_1*0+R_2*0+R_3*0+R_4*0+R_5*1}{0+0+0+0+1} = 0$
    \end{tabularx}
    \end{itemize}

    
\end{document}