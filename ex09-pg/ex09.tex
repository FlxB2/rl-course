\documentclass[a4paper]{article}
%\usepackage[ngerman]{babel}
\usepackage[T1]{fontenc}
\usepackage[utf8]{inputenc}
\usepackage{textcomp}
\usepackage{geometry}
\geometry{ left=2cm, right=2cm, top=2cm, bottom=3cm, bindingoffset=5mm}
\usepackage{graphicx}
\usepackage{xcolor}
\usepackage{hyperref}
\usepackage{longtable}
\usepackage{amstext}
\usepackage{array}
\usepackage{amsmath}
\newcolumntype{L}{>{$}l<{$}}
\usepackage{tabularx, ragged2e}
\usepackage{helvet}
\renewcommand{\familydefault}{\sfdefault}
\usepackage{lastpage}
\usepackage{todonotes}
\usepackage{titlesec}
\titleformat*{\section}{\large\bfseries}
\usepackage{listings}
\usepackage{color}

\usepackage{tikz}
%\newcommand{\tikzmark}[2]{\tikz[overlay, remember picture] \node[inner sep=0pt, outer sep=0pt, anchor=base] (#1) {#2};}
\usetikzlibrary{tikzmark}


\definecolor{mygreen}{rgb}{0.18, 0.545, 0.341}
\definecolor{mygray}{rgb}{0.5,0.5,0.5}
\definecolor{myblue}{rgb}{0.53,0.61,0.85}

\lstset{
 keywordstyle=\color{mygreen},
 commentstyle=\color{mygray},
 numbers=left,
 numbersep=5pt, 
 numberstyle=\scriptsize\color{mygray}
 }

\usepackage{amsmath,amssymb}

\DeclareRobustCommand{\bbone}{\text{\usefont{U}{bbold}{m}{n}1}}

\DeclareMathOperator{\EX}{\mathbb{E}}% expected value

\date{}
\author{}
\usepackage{fancyhdr}
\pagestyle{fancy}
\fancyhf{}
\fancyhead[R]{Felix Burk\\ Pascal Huszár}
\fancyhead[L]{Reinforcment Learning \\ Summer Term 2021 }
\fancyfoot[R]{page \thepage \text{ }/ \pageref*{LastPage}}
%\fancyfoot[LE]{Seite \thepage \text{ }von \pageref{LastPage}}
\renewcommand{\headrulewidth}{0.5pt}

\usepackage{amsmath}
\DeclareMathOperator*{\argmax}{arg\,max}
\DeclareMathOperator*{\argmin}{arg\,min}


\title{\textbf{Exercise 09}}

\begin{document}
	\maketitle 
	\thispagestyle{fancy}
	
    \section*{Task 1 - REINFORCE on the Cart-Pole}
    \subsection*{a)}
    $\frac{\delta}{\delta \theta_a}\frac{se^{s*\theta_a^T+s*\theta_b^T)}}{(e^{\theta_a^T*s}+e^{\theta_b^T*s})^2}$
 	\subsection*{b)}
 	$\pi(a|s,\theta) = \frac{e^{\theta_a^T*s}}{e^{\theta_b^T*s}+e^{\theta_c^T*s}}$ where $b,c$ are all possible actions and $a \in \{b,c\}$ \\
 	$\nabla_\theta log(\pi_\theta(s,a)) = \nabla_\theta log(\frac{e^{\theta_a^T*s}}{e^{\theta_b^T*s}+e^{\theta_c^T*s}})$ \\
 	$= \nabla_\theta log(e^{\theta_a^T*s})-\nabla_\theta log(e^{\theta_b^T*s}+e^{\theta_c^T*s})$ \\
 	$= \nabla_\theta \theta_a^T*s*log(e)-\nabla_\theta log(e^{\theta_b^T*s}+e^{\theta_c^T*s})$ \\
 	$= s -\nabla_\theta log(e^{\theta_b^T*s}+e^{\theta_c^T*s})$
 	$= s -\frac{\nabla_\theta log(e^{\theta_b^T*s}+e^{\theta_c^T*s})}{e^{\theta_b^T*s}+e^{\theta_c^T*s}}$ \\
 	$= s - \frac{e^{\theta_b^T*s}}{e^{\theta_b^T*s}+e^{\theta_c^T*s}}$ 
 	$= \frac{e^{\theta_c^T*s}*s}{e^{\theta_b^T*s}+e^{\theta_c^T*s}}$
 	$= \frac{1}{1+e^{\theta_c^T*s}}*s$ \\
 	For the case $a=c$, $a=b$ is equivalent.


\end{document}